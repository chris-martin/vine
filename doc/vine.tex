\documentclass{article}
\usepackage{fullpage,abstract,hyperref,graphicx,caption}
\usepackage{amsmath,amsfonts,amssymb,amsthm,program,float,multirow}
\hypersetup{hidelinks}

\makeatletter

\newfloat{algorithm}{thp}{lop}
\floatname{algorithm}{Algorithm}

\title{\vspace{-60pt}CS 6491 - Project 4 - Triangle Mesh Vine}

\author{
\begin{tabular}{ll}
\multirow{3}{*}{\includegraphics[scale=0.4]{chris.jpg}}\\&
Christopher Martin\\&\texttt{chris.martin@gatech.edu}\end{tabular}
~\\~\\~\\}

\newcommand\screenshot[1]{\includegraphics[scale=0.15]{#1}}
\newcommand{\argmin}{\operatornamewithlimits{argmin}}
\newcommand{\argmax}{\operatornamewithlimits{argmax}}
\def\p{\hspace{0pt}}

\begin{document}

\makeatletter
\twocolumn[
\begin{@twocolumnfalse}
\maketitle

\begin{abstract}

This project uses Scala with JOGL to
generate a tree over approximately half of the faces of
a triangle manifold
using a laced ring\cite{lr} construction,
and then render an animation resembling
a vine that grows upward along paths defined by the tree.

\end{abstract}
\end{@twocolumnfalse}
]
\makeatother

\section{Mesh structure}

The mesh is stored as a collection of components,
where each component is a collection of triangles,
and each triangle consists of three corners.
The data comes from the Stanford Bunny\cite{bunny}
read from a {\tt ply} file.
Each new triangle $t$ added to the mesh initially
belongs to its own new component.
If it is adjacent to another triangle $u$,
then the components of $t$ and $u$ are merged
into a single component.
When two components merge, one of them may have
its triangles reversed to ensure consistent corner
ordering among all of the triangles within each
component (see Figure \ref{fig:flip}).

\begin{figure}
\centering
\screenshot{mesh}
\caption{Basic rendering with mesh edges drawn.}
\end{figure}

\begin{figure}
\centering
\screenshot{flip}
\caption{When triangle C is added to this mesh,
its component is merged with A and then with B.
A and B already have compatible ordering.
B and C, however, do not, indicated by the
observation that their shared edges point in
the same direction. This is resolved by reversing
the order of the corners in C.}
\label{fig:flip}
\end{figure}

\section{Triangle forest}

The first task is to construct an undirected
acyclic graph of triangles (rendered as the
darker color in Figure \ref{fig:lr}).
The graph is initialized with a triangle
located near the bottom-center of the model
space, and is subsequently expanded by conducting
a traversal of the mesh, rejecting triangles
that share any edges with a triangle already
in the graph. A breadth-first strategy was chosen
over depth-first to exhibit more vine-like behavior.

\begin{figure}
\centering
\screenshot{cycle}
\caption{Demonstration of LR result. The darker-colored
surface indicates triangles that belong to the tree.
The resulting psuedo-Hamiltonian cycle is drawn in black.}
\label{fig:lr}
\end{figure}

The bunny is slightly sunken into the mud puddle
so that its bottommost vertices are ``underground''.
The triangle tree is split into a triangle directed forest,
where the underground nodes are used as the tree roots.
This ensures that all the vine segments sprouting from the
ground do so simultaneously at the beginning of the animation.

\section{Vine rendering}

The vine itself is rendered by using {\tt gluCylinder}
to construct each segment.
See Figure \ref{fig:segment} for discussion of how
segment endpoints are located.
The thickness of each segment is manipulated to create
a vine that is thicker at its roots and grows over time.
Thickness is calculated as $($ $c_1$ $+$
$c_2\; \text{atan($\alpha+c_3$)}$ $+$
$c_4\; \text{atan($-\beta+c_5$)}$ $) t$,
where the $c$ are constants,
$\alpha$ is the maximum distance from the node to any
leaf in the tree, $\beta$ is the distance from the node
to the root of the tree, and $t$ is the time.

\begin{figure}
\centering
\screenshot{segment}
\caption{Segments of the vine are defined over
triplets of consecutive triangles.
For example, the sequence of triangles
(ABC, BCD, CDE)
corresponds to a vine segment from the midpoint
of BC to the midpoint of CD.}
\label{fig:segment}
\end{figure}

\begin{figure}
\centering
\screenshot{partial}
\caption{In-progress animation.}
\end{figure}

\begin{figure}
\centering
\screenshot{full}
\caption{Completed animation.}
\end{figure}

\bibliography{vine}{}
\bibliographystyle{plain}

\end{document}

